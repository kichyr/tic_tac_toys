\documentclass[12pt,a4paper]{scrartcl}
\usepackage[utf8]{inputenc}
\usepackage[english,russian]{babel}
\usepackage{indentfirst}
\usepackage{tikz}
\usepackage[linguistics]{forest}
\usepackage{mathtools}
\DeclarePairedDelimiter{\ceil}{\lceil}{\rceil}

\usepackage{misccorr}
\usepackage{graphicx}
\usepackage{amsmath}
\usepackage{amsfonts}
\usepackage{amsmath}
\usepackage{pb-diagram}
\usetikzlibrary{positioning}
\usepackage{amsmath}
\usepackage[usenames,dvipsnames]{pstricks}
\usepackage{epsfig}
\usepackage{multicol}
\usepackage{pst-grad} % For gradients
\usepackage{pst-plot} % For axes
\usepackage[unicode, pdftex]{hyperref}

\begin{document}
\subsection*{Чеканов К.Ю 774}
\subsubsection*{Формулировка решенных задач.}
\textbf{1)} Решим задачу поиска кол-ва всевозможных полей в игре крестики-нолики, где под полем будем понимать состояние квадрата 3*3 в какой-то игре в какой-то ее момент (ориентация игрового поля в пространстве учитывается).\\

\textbf{2)} Решим задачу нахождения всех финальных состояний поля (состояния поля 3*3 на конец некоторой игры) не учитывая ориентацию поля в пространстве. (Промежуточная задача необходима для проверки корректности есть ответ на википедии \url{https://en.wikipedia.org/wiki/Tic-tac-toe})\\

\textbf{3)} Решим задачу поиска кол-ва всевозможных полей в игре крестики-нолики не учитывая ориентацию поля в пространстве.

\subsubsection*{Пункт 1. Комбинаторное решение.}
На каждом уровне дерева игры вычислим кол-во всевозможных состояний поля.\\

\textbf{1 уровень:} Ставим крестик в одну из клеток \textbf{9}.\\

\textbf{2 уровень:} Ставим нолик в оставшиеся 8 клеток $9\cdot8 = $\textbf{72}.\\

\textbf{3 уровень:} Ставим на поле два крестика их порядок не важен, и еще нолик $C_9^2*7 = \textbf{252}$.\\

\textbf{4 уровень:} Аналогично ставим 2 крестика, и два нолика их порядок не важен $C_9^2*C_7^2 = \textbf{756}$.\\

\textbf{5 уровень:} Ставим на поле три крестика их порядок не важен, и еще 2 нолика, начнут появлятся финальные состояние, но пока их не нужно убирать $C_9^3*C_6^2 = \textbf{1260}$.\\

\textbf{6 уровень:} Аналогично всего расстоновок $C_9^3*C_6^3$, из них $8\cdot C_6^3$ там где крестики уже встали в ряд, вычитаем их, получаем \textbf{1520}. \\

\textbf{7 уровень:} Всего расстоновок $C_9^4*C_5^3$, из них $8\cdot C_6^3$ там где нолики уже встали в ряд $C_6^4$, вычитаем их, получаем \textbf{1140}.\\

\textbf{8 уровень:} Всего расстоновок $C_9^4*C_5^4$, из них $2*8\cdot C_6^4$(можно доставить еще один крестик не входящий в тройку) там где крестики уже встали в ряд $C_6^4$, вычитаем их, получаем \textbf{390}.\\

\textbf{9 уровень:} Аналогично получаем \textbf{78}.\\

Суммируем полученные значения получаем: \textbf{5477}.

\subsubsection*{Пункт 1, 2, 3 программное решение.(Приложение: файл calculation.cpp)}

\textbf{Краткое описание алгоритма.}\\
Собственно, построим алгоритм, строящий дерево игры крестики нолики, если оценить его сверху оно имеет размер порядка $2\cdot9! \approx 7\cdot10^5$, что легко поддается вычислению. Реализуем этот алгоритм на C++, для уменьшения размера дерева будем хранить поле внутри переменной типа $int$ (кодируем каждую ячейку поля двумя байтами), реализуем соответсвующие функции и методы кодировки и декодировки, а также метод проверки поля на завершения игры. Классическим способом реализуем дерево в виде структуры значение-указатели. Далее очевидным полным перебором построим это дерево
$7C_9^2 + 2C_9^2 +9$. Имея это дерево далее анализируем его добавляя в контейнер $set$ все поля с учетом и без учета ориентации.

\textbf{Ответ.}\\
Полученные ответ совпадает с результатом вышеприведенных вычислений, а также результат, полученный для пунктв 2 совпадает со значением из википедии. На основании этого можно делать вывод о корректности работы прграммы.

\begin{center}
    \includegraphics[width=0.8\linewidth]{ttt.png}
\end{center}

\textbf{Подробнее про пункты 2, 3}.\\
Очевидно, что если не фиксировать положение доски в пространстве, то кол-во всевозможных полей на самом деле меньше. Учтем это, добавив всевозможные повороты и отражения, и получим ответ на пункты 2 и 3. 



\end{document}

